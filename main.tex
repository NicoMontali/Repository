\documentclass[a4, 12pt, twoside,openany]{book}
\usepackage[T1]{fontenc} %come sotto
\usepackage[utf8]{inputenc} %aggiunge caratteri accentati
\usepackage[italian]{babel} %lingua
\usepackage[osf]{libertinus} %font generale del documento
\usepackage{indentfirst} %rientro paragrafo
\pagestyle{plain} %nessun heading o foot particolare
\usepackage[a4paper,top=3cm,bottom=3cm,left=3cm,right=3cm]{geometry} %impaginazione e margini documento

\usepackage{graphicx, wrapfig} %gestione immagini e grafiche
\graphicspath{  {./images/}  } %cartella delle immagini

\usepackage{tocloft} %rimuove il grassetto dall'indice
\renewcommand{\cfttoctitlefont}{\huge\centerline}
\renewcommand{\cftchapfont}{\large}

\usepackage{titlesec} %cambia le impostazioni dei titoli
\titleformat{\chapter}[display]
{\normalfont\huge\filcenter}{\thechapter}{20pt}{\Huge} %ho rimosso bfseries 
\titleformat{\section}                          %dalle voci contrassegnate
{\normalfont\Large}{\thesection}{1em}{}         %%%qui
\titleformat{\subsection}[runin]
{\normalfont\large}{\thesubsection}{1em}{}      %%%qui
\titleformat{\subsubsection}[runin]
{\normalfont\normalsize\bfseries}{\thesubsubsection}{1em}{}
\titleformat{\paragraph}[runin]
{\normalfont\normalsize\bfseries}{\theparagraph}{1em}{}
\titleformat{\subparagraph}[runin]
{\normalfont\normalsize\bfseries}{\thesubparagraph}{1em}{}
\titlespacing*{\chapter}{0pt}{50pt}{40pt}
\titlespacing*{\section}{0pt}{3.5ex plus 1ex minus .2ex}{2.3ex plus .2ex}
\titlespacing*{\subsection}{0pt}{3.25ex plus 1ex minus .2ex}{1.5ex plus .2ex}
\titlespacing*{\subsubsection}{0pt}{3.25ex plus 1ex minus .2ex}{1.5ex plus .2ex}
\titlespacing*{\paragraph}{0pt}{3.25ex plus 1ex minus .2ex}{1em}
\titlespacing*{\subparagraph}{\parindent}{3.25ex plus 1ex minus .2ex}{1em}

\usepackage[nottoc]{tocbibind} %include la voce bibliografia nell'indice
\usepackage[autostyle,italian=guillemets]{csquotes} %rende più semplice la gestione di bibtex e permette inoltre di citare cose estese

\usepackage{etoolbox} %setta il carattere delle citazioni estese come più piccolo e rimuove il separatore verticale
\AtBeginEnvironment{quote}{\vspace{-\topsep}\small}
\AtEndEnvironment{quote}{\vspace{-\topsep}}

\usepackage[backend=biber,bibstyle=authortitle,citestyle=verbose-trad1]{biblatex}
\addbibresource{bibliography.bib}
\renewcommand\mkbibnamefamily[1]{\textsc{#1}} %autori in maiuscoletto

\usepackage{hyperref} %NB da caricare per ultimo, nel caso usare \phantomsection per la bibliografia
%%%%%%%%%%%%%%%%%%%%%%%%%%%%%%%%%%%%%%%%%%%%%%%%%%%%%%%%%%%%%%%%%
\begin{document}
\begin{titlepage}
    \begin{figure}
            \includegraphics[scale=0.45]{images/logosns.png}
            \flushleft
    \end{figure}
    
    \begin{center}  
        \vspace{0.5cm}
        \Huge
        \textsc{Pinco}
            
        \vspace{0.1cm}
        \LARGE
        \textsc{Pallino}
        \vspace{1.5cm}
        
        \Large
        \textsc{Harambe}
        \vspace{2.5cm}
        
        \large
        \textsc{Classe di Fuffosofia}\\
        \textsc{Scuola Normale Superiore}\\
        \textsc{Pisa, a.a. 2kx/2kx}\\
        \vfill
    \end{center}    
    \flushright
    \Large
    Rel.: Prof. Ciccio pasticcio\\
    Colloquio di passaggio d'anno
    
    \vspace{0.8cm}
    
\end{titlepage} %inserisce il frontespizio
%\thispagestyle{empty} %sospende la numerazione della pagina
\phantomsection %inserisce l'indice nell'indice
\addcontentsline{toc}{chapter}{Indice}
\tableofcontents %SOMMARIO
%\thispagestyle{empty}

\chapter*{Introduzione} %sospende la numerazione dei capitoli per la sezione
\addcontentsline{toc}{chapter}{Introduzione}
\begin{titlepage} %crea l'enviroment
\begin{figure}[t] %inserisce le figure
	\centering\includegraphics[width=0.9\textwidth]{testimages/scritta.png}
    \centering\includegraphics[width=0.4\textwidth]{testimages/logo.png}
\end{figure}
\begin{center}
	\textbf{ Dipartimento di LOL\\ Corso di Laurea in TeXXaggio avanzato\\}
	\vspace{15mm}
    {\LARGE{\bf Primo Titolo}}\\
	\vspace{3mm}
	{\LARGE{\bf Secondo Titolo}}\\
\end{center}

\vspace{36mm}
%minipage divide la pagina in due sezioni settabili
\begin{minipage}[t]{0.47\textwidth}
	{\large{\bf Relatori:\\ Prof. Pinco\\ Prof. Pallino}}
\end{minipage}
\hfill
\begin{minipage}[t]{0.47\textwidth}\raggedleft
	{\large{\bf Presentata da: \\ Ciccio Pasticcio\\ }}
\end{minipage}

\vspace{18mm}

\centering{\large{\bf Sessione estiva\\ Anno Accademico 2kxx/2kxy }}

\end{titlepage} %input permette il nesting, \include lo impedisce

\chapter{Capitolo 1}
%\input{}

\chapter{Capitolo 2}
%\input{}

\chapter{Capitolo 3}

%\bibliographystyle{siam} %questo set di comandi esclude e rimpiazza DEL TUTTO i comandi definiti come da preambolo. In caso di modifica per tornare ai comandi standard, verificare di sostituire correttamente TUTTI i componenti, sia nel preambolo che nella coda del documento.
%\bibliography{bibliography}
\printbibliography %inserisce le voci citate
\end{document}