\documentclass[a4paper, 12pt, twoside,openany]{book}
\usepackage[T1]{fontenc} %come sotto
\usepackage[utf8]{inputenc} %aggiunge caratteri accentati
\usepackage[italian]{babel} %lingua
\usepackage[osf]{libertinus} %font generale del documento
\usepackage{indentfirst} %rientro paragrafo
\pagestyle{plain} %nessun heading o foot particolare
\usepackage[a4paper,top=3cm,bottom=3cm,left=3cm,right=3cm]{geometry} %impaginazione e margini documento
%\usepackage{layaureo} %alternative layout, NECESSARIO SPECIFICARE A4PAPER IN DOCCLASS
%GESTIONE IMMAGINI E GRAFICHE
\usepackage{graphicx, wrapfig} %gestione immagini e grafiche
\graphicspath{  {./testimages/}  } %cartella delle immagini
%GESTIONE TITOLI NEL SOMMARIO
\usepackage{tocloft} %rimuove il grassetto dall'indice
\renewcommand{\cfttoctitlefont}{\huge\centerline}
\renewcommand{\cftchapfont}{\large}
%GESTIONE TITOLI PARAGRAFI E SEZIONI
\usepackage{titlesec} %cambia le impostazioni di default dei titoli delle sezioni
%NB queste impostazioni rispettano le linee guida d'ateneo
%NB no punto finale nei titoli delle sezioni del corpo del testo
\titleformat{\chapter}[display] %inserire "\chaptertitlename" per ripristinare la dicitura "capitolo" sopra il titolo del capitolo stesso.
{\normalfont\huge\bfseries\scshape}{\thechapter}{20pt}{\Huge} 
\titleformat{\section}
{\normalfont\Large\bfseries}{\thesection}{1em}{}
\titleformat{\subsection}[runin] %runin fa del titolo un elemento inline
{\normalfont\large\itshape}{\normalfont\thesubsection}{1em}{}
\titleformat{\subsubsection}[runin]
{\normalfont\normalsize\bfseries}{\thesubsubsection}{1em}{}
\titleformat{\paragraph}[runin]
{\normalfont\normalsize\bfseries}{\theparagraph}{1em}{}
\titleformat{\subparagraph}[runin]
{\normalfont\normalsize\bfseries}{\thesubparagraph}{1em}{}
\titlespacing*{\chapter}{0pt}{50pt}{40pt}
\titlespacing*{\section}{0pt}{3.5ex plus 1ex minus .2ex}{2.3ex plus .2ex}
\titlespacing*{\subsection}{0pt}{3.25ex plus 1ex minus .2ex}{1.5ex plus .2ex}
\titlespacing*{\subsubsection}{0pt}{3.25ex plus 1ex minus .2ex}{1.5ex plus .2ex}
\titlespacing*{\paragraph}{0pt}{3.25ex plus 1ex minus .2ex}{1em}
\titlespacing*{\subparagraph}{\parindent}{3.25ex plus 1ex minus .2ex}{1em}
%GESTIONE ASPETTO DELLE CITAZIONI ESTESE
\usepackage[nottoc]{tocbibind} %include la voce bibliografia nell'indice
\usepackage[autostyle,italian=guillemets]{csquotes} %rende più semplice la gestione di bibtex e permette inoltre di citare cose estese
%SAME
\usepackage{etoolbox} %setta il carattere delle citazioni estese come più piccolo e rimuove il separatore verticale
\AtBeginEnvironment{quote}{\vspace{-\topsep}\small}
\AtEndEnvironment{quote}{\vspace{-\topsep}}
%GESTIONE BIBLIOGRAFIA
%\usepackage{biblatex-chicago} %ottimo pacchetto, molto ricco ma anche complicato da settare
\usepackage[backend=biber,bibstyle=authortitle, giveninits=true,citestyle=verbose-trad1]{biblatex}
\addbibresource{bibliography.bib} %specifica il file della bibliografia
\renewcommand*{\newunitpunct}{\addcomma\space} %separa i campi con virgole e non con punti, come di default
\renewcommand*{\revsdnamepunct}{} %rimuove la virgola tra nome e cognome dell'autore
\renewcommand\mkbibnamefamily[1]{\textsc{#1}} %autori in maiuscoletto
%LINK: INTERNI ED ESTERNI AL DOCUMENTO
%\usepackage{hyperref} %NB da caricare per ultimo, nel caso usare \phantomsection per la bibliografia
%%%%%%%%%%%%%%%%%%%%%%%%%%%%%%%%%%%%%%%%%%%%%%%%%%%%%%%%%%%%%%%%%
%%%%%%%%%%%%%%%%%%%%%%%%%%%%%%%%%%%%%%%%%%%%%%%%%%%%%%%%%%%%%%%%%
\begin{document}
\input{titlepageSNS} %inserisce il frontespizio
\begin{titlepage} %crea l'enviroment
\begin{figure}[t] %inserisce le figure
	\centering\includegraphics[width=0.9\textwidth]{testimages/scritta.png}
    \centering\includegraphics[width=0.4\textwidth]{testimages/logo.png}
\end{figure}
\begin{center}
	\textbf{ Dipartimento di LOL\\ Corso di Laurea in TeXXaggio avanzato\\}
	\vspace{15mm}
    {\LARGE{\bf Primo Titolo}}\\
	\vspace{3mm}
	{\LARGE{\bf Secondo Titolo}}\\
\end{center}

\vspace{36mm}
%minipage divide la pagina in due sezioni settabili
\begin{minipage}[t]{0.47\textwidth}
	{\large{\bf Relatori:\\ Prof. Pinco\\ Prof. Pallino}}
\end{minipage}
\hfill
\begin{minipage}[t]{0.47\textwidth}\raggedleft
	{\large{\bf Presentata da: \\ Ciccio Pasticcio\\ }}
\end{minipage}

\vspace{18mm}

\centering{\large{\bf Sessione estiva\\ Anno Accademico 2kxx/2kxy }}

\end{titlepage} %input permette il nesting, \include lo impedisce
%\thispagestyle{empty} %sospende la numerazione della pagina
\phantomsection %evita ridondanze con hyperref
\addcontentsline{toc}{chapter}{Indice} %inserisce l'indice nell'indice
\tableofcontents %SOMMARIO
%\thispagestyle{empty} %come sopra

\chapter*{Introduzione} %sospende la numerazione dei capitoli per la sezione
\addcontentsline{toc}{chapter}{Introduzione}
Il presente testo è brutalmente copypastato dalle norme redazionali disponibili sul sito UniPi (per il link diretto, vd. README.md)

Il titolo della tesi, di solito non piu` lungo di due righe,
deve fornire un’idea precisa del contenuto, “per facilitare l’elaborazione di
elenchi di titoli, l’indicizzazione e il reperimento dell’informazione” (UNI
ISO 7144, par. 7.3.1).
2 RIASSUNTO
E` una sezione molto importante. Presenta il lavoro fatto e i risultati ottenuti.
Va posto al centro della seconda pagina. Non dovrebbe superare le 200 parole
e in nessun caso la pagina stessa.
3 INDICE
Deve riportare i capitoli, le sezioni e le sottosezioni del testo, utilizzandone
la stessa numerazione, le stesse pagine e le stesse parole. L’organizzazione
dell’indice deve riflettere quella del testo, anche in senso spaziale: pertanto
se nel testo la sezione 1.2 `e una suddivisione del capitolo 1, questo dovr` a
risultare evidente anche in termini di allineamento. E` allegato un esempio.
4 INTRODUZIONE
Di solito `e organizzata in tre sezioni. La prima, “Presentazione del problema”, introduce il lettore al problema affrontato descrivendo con chiarezza
l’ambito in cui si colloca e la sua importanza. La seconda, “Rassegna della
letteratura”, riassume i lavori piu` significativi consultati ed evidenzia i loro
contributi specifici alla soluzione degli aspetti del problema che verranno
approfonditi nella tesi. La terza, infine, “Contenuto della tesi”, presenta gli
obiettivi della tesi e, in poche righe, il contenuto dei capitoli con lo scopo
di anticipare, in modo chiaro e conciso, ci` o che `e stato fatto, perch´e e con quali risultati. Se l’introduzione risulta breve, la suddivisione in sezioni si
puo` omettere.
5 CORPO DELLA TESI
Va organizzato in capitoli, sezioni e sottosezioni (con la numerazione che
non termina con il punto) ognuno con il proprio titolo allineato a sinistra,
nei seguenti formati:
1 PER I CAPITOLI USARE CARATTERI MAIUSCOLI IN
GRASSETTO
1.1 Per le sezioni principali usare caratteri minuscoli in grassetto
1.1.1 Per le sezioni secondarie usare caratteri minuscoli in corsivo
Per le sezioni non numerate usare caratteri minuscoli in grassetto
oppure
Caratteri minuscoli in corsivo
in quanto la differenza nei caratteri `e sufficiente a creare uno stacco.
Non `e indispensabile la presenza di tutte le sopracitate sezioni.
Ogni capitolo dovrebbe iniziare con un breve riassunto del suo contenuto
e terminare con una sezione conclusiva che riepiloga gli aspetti trattati.
La grandezza dei caratteri usati per numerare le pagine e gli indici nelle
notazioni matematiche deve essere inferiore a quella usata per il testo.
Evidenziare l’inizio dei paragrafi con il rientro della riga o con una riga
bianca, ma una volta fatta una scelta uniformarsi a questa.
“Il testo va dattiloscritto a spazio due e la prima parola del paragrafo
rientra di tre battute. Si pu` o decidere di rientrare di tre spazi solo all’inizio
del paragrafo o all’inizio di ogni capoverso (e cio`e ad ogni a capo), come
stiamo facendo in questa pagina.
Il rientro dopo l’a capo `e importante perch´e permette di capire subito
che il capoverso precedente si `e concluso e che il discorso riprende dopo una
pausa. Come abbiamo gi` a visto `e bene andare a capo sovente, ma non si
deve andare a capo a caso. L’a capo significa che un periodo filato, composto
di varie frasi, si `e organicamente concluso e che inizia un’altra porzione del
discorso. E` come se parlando a un certo punto ci interrompessimo per dire
2“Capito? D’accordo? Bene, allora proseguiamo”. Una volta che tutti sono
d’accordo si va a capo e si prosegue, esattamente come stiamo facendo in
questo caso”. 2
Fare un uso discreto degli stili; in particolare sono da evitare le stravaganze come caratteri colorati, bordati e ombreggiati, che rendono di solito
il testo illeggibile perch´e sono previsti per fonti molto grandi, e utilizzare
poche fonti: una per il testo (ad es. Times), eventualmente una diversa per
i titoli (ad es. Helvetica), una per i simboli matematici (ad es. Symbol) e una
per simulare la scrittura di macchine da scrivere per inserire ad esempio programmi all’interno di un testo (ad es. Courier, fonte a spaziatura fissa
(12 punti), da usare nel testo con fonte di dimensione inferiore (10
punti), e non proporzionale).
Per la stesura della tesi Eco commenta, in modo interessante, diversi
aspetti dei quali riportiamo solo alcune affermazioni:
“La tesi `e un lavoro che per ragioni occasionali `e diretto solo al relatore o
al correlatore, ma che di fatto presume di essere letto e consultato da molti
altri, anche da studiosi non direttamente versati in quella disciplina.
... Quindi, come regola generale definire tutti i termini tecnici usati come
categorie chiavi del nostro discorso, a meno che non siano termini canonici
e indiscussi della disciplina in oggetto.”
“Non siete Proust. Non fate periodi lunghi. Se vi vengono, fateli, ma
poi spezzateli. Non abbiate paura a ripetere due volte il soggetto, lasciate
perdere troppi pronomi e subordinate”.
“Andate sovente a capo”.
“Scrivete tutto quello che vi passa per la testa, ma solo in prima stesura.
Dopo vi accorgerete che l’enfasi vi ha preso la mano e vi ha allontanato
dal centro del vostro argomento. Allora toglierete le parti parentetiche, le
divagazioni, e le metterete in nota o in appendice. La tesi serve a dimostrare
una ipotesi che avete elaborato dall’inizio, non a mostrare che voi sapete
tutto”.
“Usate il relatore come cavia. Dovete fare in modo che il relatore legga i
primi capitoli (e poi a mano a mano tutto il resto) con molto anticipo sulla
consegna dell’elaborato. Le sue reazioni potranno servirvi. Se il relatore `e
occupato (o pigro) usate un amico. Controllate se qualcuno capisce quello
che scrivete. Non giocate al genio solitario”.
“Non ostinatevi a iniziare col primo capitolo” %input permette il nesting, \include lo impedisce

\chapter{Qui 1}
%\input{}

\chapter{Quo 2}
%\input{}

\chapter{Qua 3}
%\input{}

%\bibliographystyle{siam} %questo set di comandi esclude e rimpiazza DEL TUTTO i comandi definiti come da preambolo. In caso di modifica per tornare ai comandi standard, verificare di sostituire correttamente TUTTI i componenti, sia nel preambolo che nella coda del documento.
%\bibliography{bibliography}
%ALTRA ALTERNATIVA: \usepackage{biblatex-chicago}
\printbibliography %inserisce le voci citate
\end{document}